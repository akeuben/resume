%%%%%%%%%%%%%%%%%
% This is an sample CV template created using altacv.cls
% (v1.7, 9 August 2023) written by LianTze Lim (liantze@gmail.com). Compiles with pdfLaTeX, XeLaTeX and LuaLaTeX.
%
%% It may be distributed and/or modified under the
%% conditions of the LaTeX Project Public License, either version 1.3
%% of this license or (at your option) any later version.
%% The latest version of this license is in
%%    http://www.latex-project.org/lppl.txt
%% and version 1.3 or later is part of all distributions of LaTeX
%% version 2003/12/01 or later.
%%%%%%%%%%%%%%%%

%% Use the "normalphoto" option if you want a normal photo instead of cropped to a circle
% \documentclass[10pt,a4paper,normalphoto]{altacv}

\documentclass[10pt,a4paper,ragged2e,withhyper]{altacv}
%% AltaCV uses the fontawesome5 and packages.
%% See http://texdoc.net/pkg/fontawesome5 for full list of symbols.

% Change the page layout if you need to
\geometry{left=1.25cm,right=1.25cm,top=1.5cm,bottom=1.5cm,columnsep=1.2cm}

% The paracol package lets you typeset columns of text in parallel
\usepackage{paracol}
\usepackage[dvipsnames]{xcolor}

% Change the font if you want to, depending on whether
% you're using pdflatex or xelatex/lualatex
% WHEN COMPILING WITH XELATEX PLEASE USE
% xelatex -shell-escape -output-driver="xdvipdfmx -z 0" sample.tex
\ifxetexorluatex
  % If using xelatex or lualatex:
  \setmainfont{Roboto Slab}
  \setsansfont{Lato}
  \renewcommand{\familydefault}{\sfdefault}
\else
  % If using pdflatex:
  \usepackage[rm]{roboto}
  \usepackage[defaultsans]{lato}
  % \usepackage{sourcesanspro}
  \renewcommand{\familydefault}{\sfdefault}
\fi

% Change the colours if you want to
\definecolor{SlateGrey}{HTML}{000d19}
\definecolor{LightGrey}{HTML}{666666}
\definecolor{DarkPastelRed}{HTML}{016ACE}
\definecolor{PastelRed}{HTML}{4da9fe}
\definecolor{GoldenEarth}{HTML}{013565}
\colorlet{name}{black}
\colorlet{tagline}{PastelRed}
\colorlet{heading}{DarkPastelRed}
\colorlet{headingrule}{GoldenEarth}
\colorlet{subheading}{PastelRed}
\colorlet{accent}{PastelRed}
\colorlet{emphasis}{SlateGrey}
\colorlet{body}{LightGrey}

% Change some fonts, if necessary
\renewcommand{\namefont}{\Huge\rmfamily\bfseries}
\renewcommand{\personalinfofont}{\footnotesize}
\renewcommand{\cvsectionfont}{\LARGE\rmfamily\bfseries}
\renewcommand{\cvsubsectionfont}{\large\bfseries}


% Change the bullets for itemize and rating marker
% for \cvskill if you want to
\renewcommand{\cvItemMarker}{{\small\textbullet}}
\renewcommand{\cvRatingMarker}{\faCircle}
% ...and the markers for the date/location for \cvevent
% \renewcommand{\cvDateMarker}{\faCalendar*[regular]}
% \renewcommand{\cvLocationMarker}{\faMapMarker*}


% If your CV/résumé is in a language other than English,
% then you probably want to change these so that when you
% copy-paste from the PDF or run pdftotext, the location
% and date marker icons for \cvevent will paste as correct
% translations. For example Spanish:
% \renewcommand{\locationname}{Ubicación}
% \renewcommand{\datename}{Fecha}


%% Use (and optionally edit if necessary) this .tex if you
%% want to use an author-year reference style like APA(6)
%% for your publication list
% \input{pubs-authoryear.tex}

%% Use (and optionally edit if necessary) this .tex if you
%% want an originally numerical reference style like IEEE
%% for your publication list
\input{pubs-num.tex}

%% sample.bib contains your publications
\addbibresource{sample.bib}

\begin{document}
\name{Avery Keuben}
\tagline{Software Engineer}
%% You can add multiple photos on the left or right
\photoR{2.8cm}{photo}
% \photoL{2.5cm}{Yacht_High,Suitcase_High}

\personalinfo{%
  % Not all of these are required!
  \email{avery1516@gmail.com}
  \location{Canada}
  \homepage{kappabyte.github.io}
  \github{Kappabyte}
  %% You can add your own arbitrary detail with
  %% \printinfo{symbol}{detail}[optional hyperlink prefix]
  % \printinfo{\faPaw}{Hey ho!}[https://example.com/]

  %% Or you can declare your own field with
  %% \NewInfoFiled{fieldname}{symbol}[optional hyperlink prefix] and use it:
  % \NewInfoField{gitlab}{\faGitlab}[https://gitlab.com/]
  % \gitlab{your_id}
  %%
  %% For services and platforms like Mastodon where there isn't a
  %% straightforward relation between the user ID/nickname and the hyperlink,
  %% you can use \printinfo directly e.g.
  % \printinfo{\faMastodon}{@username@instace}[https://instance.url/@username]
  %% But if you absolutely want to create new dedicated info fields for
  %% such platforms, then use \NewInfoField* with a star:
  % \NewInfoField*{mastodon}{\faMastodon}
  %% then you can use \mastodon, with TWO arguments where the 2nd argument is
  %% the full hyperlink.
  % \mastodon{@username@instance}{https://instance.url/@username}
}

\makecvheader
%% Depending on your tastes, you may want to make fonts of itemize environments slightly smaller
% \AtBeginEnvironment{itemize}{\small}

%% Set the left/right column width ratio to 6:4.
\columnratio{0.6}

% Start a 2-column paracol. Both the left and right columns will automatically
% break across pages if things get too long.
\begin{paracol}{2}
\cvsection{Experience}

\cvevent{Research Assistant}{University of Calgary}{April 2023 -- Ongoing}{Calgary, Alberta}
\begin{itemize}
\item Developer for the GEARS game
\item Using React, Typescript, and Spring Boot
\end{itemize}

\divider

\cvevent{Customer Service Cashier}{Calgary Co-op}{July 2021 -- Ongoing}{Calgary, Alberta}
\begin{itemize}
\item Assist customers with transactions, membership details, and lottery sales.
\item Balance store's registers, lottery sales, and promotional item sales.
\end{itemize}

\cvsection{Projects}

\cvevent{GEARS}{Gearing up for Education, Achieving Real-world Success}{}{University of Calgary}
\begin{itemize}
\item Web based point and click adventure game
\item Helping high school students with learning disabilities and ADHD transition to postsecondary
\end{itemize}

\divider

\cvevent{KappaEngine}{Simple LWJGL Game Engine}{}{Personal Project}
\begin{itemize}
\item Simple game engine in Java using LWJGL OpenGL bindings to learn graphics programming
\item Created simple minecraft clone using the engine in a few weeks.
\end{itemize}

\medskip

\cvsection{Education}

\cvevent{B.Sc Computer Science}{University of Calgary}{Sept 2022 -- Ongoing}{}

\divider

\cvevent{High School Diploma}{Calgary Board of Education}{June 2022}{}

%% Switch to the right column. This will now automatically move to the second
%% page if the content is too long.
\switchcolumn

\cvsection{Most Proud of}

\cvachievement{\faGraduationCap}{4.00 GPA}{in first year computer science courses}

\divider

\cvachievement{\faCode}{Custom Desktop Shell}{Custom desktop shell using AGS and GTK for the Hyprland window manager}

\divider

\cvachievement{\faMusic}{WAMSB World Champion}{Got first place in the world marching band competition with the Stampede Showband}

\cvsection{Strengths}

\cvtag{Hard-working}
\cvtag{Eye for detail}\\
\cvtag{Patient}

\divider\smallskip

\cvtag{Frontend Development}
\cvtag{Backend Development}\\
\cvtag{Mathematics}

\cvsection{Programming \\[.2cm] Languages}

\cvskill{Typescript}{5}
\divider

\cvskill{Java}{4}
\divider

\cvskill{C}{3.5} %% Supports X.5 values.

\cvsection{References}

\textcolor{red}{Need to ask before publishing...}

\cvref{Prof.\ Richard Zhao Ph.D}{University of Calgary}{richard.zhao1@ucalgary.ca}
{}

\divider

\textcolor{red}{Need to ask before publishing...}

\cvref{Prof.\ Meadow Schroeder Ph.D}{University of Calgary}{schroedm@ucalgary.ca}
{}

\divider

\textcolor{red}{Need to ask before publishing...}

\cvref{Damon Smith}{Manager, Calgary Co-op}{dsmith@calgarycoop.com}
{}

%% Yeah I didn't spend too much time making all the
%% spacing consistent... sorry. Use \smallskip, \medskip,
%% \bigskip, \vspace etc to make adjustments.
\medskip


\end{paracol}


\end{document}
